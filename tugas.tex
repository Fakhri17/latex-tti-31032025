\documentclass[12pt, a4paper]{article}
\usepackage{times}
\usepackage{lipsum}
\usepackage{ragged2e}
\usepackage{blindtext}
\usepackage{longtable}
\setlength{\parindent}{1cm} % Default is 15pt.

\title{\textbf{Peran \textit{Big Data Analytics},\textit{Machine Learning}, dan \textit{Artificial Intelligence} dalam Pendeteksian \textit{Financial Fraud: A SYSTEMATIC LITERATURE REVIEW}}}
\author{
    \textbf{Finecia Shinta Dewi, Totok Dewayanto$^{1}$} \\
    \small Departemen Akuntansi, Fakultas Ekonomika dan Bisnis, Universitas Diponegoro \\
    \small Jl. Prof. Soedharto SH, Tembalang, Semarang 50239, Indonesia \\
    \small Phone: +62 24 76486851
}
\date{}

\begin{document}
\maketitle

\section*{\centering ABSTRACT}
\textit{This research aims to explore the critical role of big data analytics, machine learning, and artificial intelligence in detecting financial fraud within financial institutions. The research is based on published research articles.}

\textit{Utilizing a Systematic Literature Review with the PRISMA protocol, an analysis was conducted on 20 articles published between 2020 and 2024, sourced from the Scopus database. The findings were categorized into three areas: the role of big data analytics in financial fraud detection, the role of machine learning in financial fraud detection, and the role of artificial intelligence in financial fraud detection.}

\textit{The results indicated that financial fraud detection systems employing big data analytics (BDA) demonstrated a significant average strength (76.67\%), particularly in detection effectiveness, accuracy, and data processing speed. The implementation of artificial intelligence (AI) in detection also showed significant strength scores. In contrast to BDA and AI, some machine learning algorithms exhibited substantial weaknesses. Addressing these weaknesses in financial fraud detection at financial institutions, future research on the integration of machine learning algorithms is deemed crucial.}\\

\textit{Keywords: artificial intelligence, big data analytics, machine learning, financial fraud, SLR.}

\section*{PENDAHULUAN}

\hspace{1cm}Perjalanan Teknologi dan pengaruhnya terhadap masyrakat telah mengalami transformasi mendalam sejak revolusi industri 1.0 hingga \textit{society 5.0}.
Memasuki era \textit{society 5.0}, muncul pengintgrasian teknologi mutakhir seperti AI dan \textit{big data} ke dalam tatanan masyarakat untuk 
meningkatkan kualitas hidup manusia serta menciptakan harmoni antara kemajuan teknologi dan kesejahteraan sosial (Alimohammadlou \& Khoshsepehr, 2023). 
Integrasi ini memicu transformasi yang signifikan seperti mengubah fundamental cara lembaga keuangan dan para profesional mengelola dan melaporkan data keuangannya.

Menurut Broby (2021), kemajuan dalam teknologi keuangan telah memberikan dampak yang signifikan terhadap peningkatan efisiensi proses keuangan, 
termasuk otomatisasi tugas operasional, analisis data yang lebih komprehensif, peningkatan kapasitas penyimpanan data, dan integrasi perangkat lunak 
yang lebih efisien. Semakin efisien, semakin banyak celah keamanan yang muncul, menjadi pendorong utama bagi peningkatan kecurangan keuangan 
atau disebut dengan \textit{“financial fraud”}. Dengan kehadiran otomatisasi yang semakin maju, seperti yang disoroti oleh Anna Kooi, pemimpin layanan keuangan nasional di Wipfli LLP, 
pelaku kejahatan dapat memanfaatkan perangkat lunak dan bot untuk melakukan kecurangan dengan frekuensi yang lebih tinggi dan dalam skala yang lebih luas daripada sebelumnya (Kuehner-Hebert, 2024).

Peningkatan \textit{financial fraud} dapat berdampak buruk bagi sebuah perusahaan dan lembaga keuangan, maka diperlukan pengadopsian pendekatan proaktif untuk mendeteksi kecurangan. 
Pendekatan ini harus mengimplentasikan teknologi terkini untuk menganalisis data secara mendalam dan mengidentifikasi pola yang mencurigakan. 
Salah satu solusi yang semakin diandalkan dalam upaya mendeteksi kecurangan keuangan adalah analitik data besar atau dikenal dengan sebutan \textit{“big data analytics”}. 
Dengan menggunakan algoritma dan teknik analisis yang canggih, analitik data memungkinkan perusahaan untuk menggali informasi berharga dari volume besar 
data transaksi keuangan dan perilaku pelaku bisnis (Kesuma \& Sunarto, 2020).

Seiring berkembangnya tren dan pola kecurangan, pengoptimalan pendeteksian menjadi hal yang sangat esensial.
Gupta  (2024) menemukan  bahwa pemanfaatan \textit{machine learning}, \textit{data analytics}, dan sistem \textit{artificial intelligence} dapat meningkatkan deteksi transaksi kecurangan dengan presisi yang lebih besar. 
Dengan memanfaatkan solusi \textit{big data} dan teknik \textit{machine learning}, deteksi kecurangan dapat  dilakukan  secara real-timedengan  tingkat  akurasi  yang  tinggi (Abbassi  et  al., 2021).

Dengan Demikian, berikut adalah pertanyaan penelitian dalam penelitian ini:\\
RQ1: Apakah \textit{big data analytics, machine learning,} dan \textit{artificial intelligence} memiliki peran penting dalam pendeteksian \textit{financial fraud}di lembaga keuangan?\\
RQ2: Bagaimana  perkembangan  penelitian  tentang \textit{big data analytics, machine learning,} dan \textit{artificial intelligence}?\\
RQ3: Apa arah penelitian  yang  dapat  dieksplorasi  di  masa  mendatang  dalam  usaha  pendeteksian \textit{financial fraud}?

Penelitian ini bertujuan untuk melakukan \textit{systematic literature review} (SLR) mengenaipentingnya peran \textit{big data analytics, machine learning,} dan \textit{artificial intelligence} 
dalam pendeteksian \textit{financial fraud} di lembaga keuangan serta mengeksplorasi dan meninjau variasi hasil temuan yang terdapat dalam   penelitian   empiris   terkait   peran
\textit{big data analytics, machine learning,} dan \textit{artificial intelligence} dalam pendeteksian \textit{financial fraud} di lembaga keuangan dan melakukan
sintesispada temuan yang didapat.

\section*{TINJAUAN PUSTAKA DAN KERANGKA PEMIKIRAN TEORITIS}
\hspace{1cm}Bagian  ini menjelaskan  teoridan  konsep sertakerangka  pemikiran yang  digunakan  dalam penelitian.\\

\subsection*{\textit{Data Science Theory}}
\hspace{1cm}\textit{Data science} adalah bidang interdisipliner yang mengeksplorasi proses pengumpulan, 
analisis, interpretasi, dan pemodelan data untuk mendapatkan wawasan yang berharga.

\section*{METODE PENELITIAN}
\hspace{1cm}Bagian ini menjelaskan perumusan pertanyaan penelitian, strategi pencarian literatur, ktriteria literatur, 
dan seleksi literatur.

\subsection*{Perumusan Pertanyaan Penelitian \textit{(Research Question)}}
\hspace{1cm}Sebagai acuan dalam merumuskan pertanyaan penelitian, diadopsi kerangka kerja 
\textit{population, intervention, comparasion} dan \textit{outcome} (PICO) untuk memperoleh kata kunci yang mempermudah proses pencarian.

\begin{center}
    \renewcommand{\arraystretch}{1.5} % Adjust row spacing
    \subsubsection*{Tabel 1}
    \subsubsection*{Framework PICO}
    \begin{longtable}{|l|l|}
        \hline
        \textbf{PICO \textit{Tool}} & \\ \hline
        Population         & Lembaga Keuangan \\ \hline
        Intervention       & \textit{Big data analytics, machine learning,} dan \textit{artificial intelligence} \\ \hline
        Comparison         & - \\ \hline
        Outcome            & Pendeteksian \textit{Financial Fraud} \\ \hline
    \end{longtable}
\end{center}

\begin{flushleft}
Sumber: Data Diolah, 2024
\end{flushleft}

Berdasarkan kerangka PICO yang dirumuskan tersebut, kata kunci yang digunakan adalah \textit{big data analytics, machine learning,} 
dan \textit{artificial intelligence; financial fraud detection;} dan \textit{financial institutions}.
Kata kunci tersebut merupakan kata kunci dasar yang akan dikembangkan saat melakukan pencarian literatur.

\subsection*{Strategi Pencarian Literatur}
\hspace{1cm}Sebelum memulai pencarian, diperlukan pemilihan database yang sesuai untuk meningkatkan kemungkinan menemukan artikel
yang sangat relevan. Database yang dipilih adalah Scopus, sehingga data yang dipakai dalam penelitian adalah data sekunder.
Hasil-hasil penelitian dalam jurnal yang dipublikasikan secara \textit{online} merupakan data tersebut.
Untuk mencari artikel yang relevan pada Scopus, perlu mendefinisikan \textit{search string} terlebih dahulu.
Berikut merupakan langkah-langkah yang digunakan untuk membuat \textit{search string} pada Scopus:

\begin{enumerate}
    \setlength{\itemsep}{0pt}
    \item Mengembangkan  kata/istilah  pencarian  dari  PICO,  terutama  dari \textit{population, intervention,} dan \textit{outcome}
    \item Mengembangkan kata/istilah pencarian dengan kata kunci yang relevan
    \item Mengembangkan sinonim dan ejaan alternatif dari kata/istilah pencarian
    \item Masukkan \textit{search string} ke dalam mesin pencarian Scopus.
    \item Mengkombinasikan \textit{search string} menggunakan   kata/istilah   pencarian   yang   telah dikembangkan dengan menggunakan 
    operator \textit{boolean} (AND dan OR).
\end{enumerate}

Search string yang akhirnya digunakan adalah 
\textit{(("big data" OR "big data analy*" OR "artificial intelligence" OR "machine learning" OR "deep learning") 
AND (fraud OR "fraud detection" OR "financial fraud") AND ("financial institutions" OR "financial industry" 
OR "bank" OR "insurance" OR corporate OR fintech))}.

\subsection*{Kriteria Literatur}
\hspace{1cm}Kriteria literatur dibagi menjadi kriteria inklusi dan eksklusi. Kriteria pada penelitian ini disesuaikan
dengan \textit{framework} PICO yang sebelumnya telah dibuat dengan tambahan sebagai berikut

\begin{center}
    \renewcommand{\arraystretch}{1.5} % Adjust row spacing
    \subsubsection*{Tabel 2}
    \subsubsection*{Kriteria Inklusi dan Eksklusi}
    \begin{longtable}{|p{2cm}|p{4cm}|p{4cm}|}
        \hline
        \textbf{Kriteria} & \textbf{Inklusi} & \textbf{Eksklusi} \\ 
        \hline
        Subjek & 
        Topik penelitian berkaitan dengan pendeteksian \textit{financial fraud} & 
        Topik penelitian tidak berkaitan dengan pendeteksian \textit{financial fraud} \\ 
        \hline
        Bahasa &
        Bahasa Inggris &
        Bahasa selain Inggris \\
        \hline
        Sumber &
        Artikel penelitian yang \textit{open access} &
        Artikel penelitian yang tidak \textit{open access} \\
        \hline
        Jangka Waktu &
        Tahun 2020-2024 &
        Sebelum tahun 2020-2024 \\
        \hline
        Tema isi Jurnal &
        Membahas mengenai peran \textit{big data analytics, machine learning,} dan \textit{artificial intelligence} dalam pendeteksian \textit{financial fraud} &
        Artikel tidak sesuai dengan \textit{research question} setelah dilakukan analisis secara detail \\
        \hline
    \end{longtable}
\end{center}

\begin{flushleft}
Sumber: Data Diolah, 2024
\end{flushleft}

\subsection*{Seleksi Literatur}
\hspace{1cm}Data dikumpulkan melalui database Scopus dan dianalisis menggunakan kerangka kerja PRISMA
(\textit{Preferred Reporting Items for Systematic Reviews and Meta-Analyses}) yang terdiri dari 3 tahapan sebagai berikut.

\begin{center}
    \subsubsection*{Gambar 3}
    \subsubsection*{PRISMA \textit{Flow Diagram}}
    <Riyan>
\end{center}

\begin{flushleft}
Sumber: Data Diolah, 2024
\end{flushleft}

\section*{HASIL PENELITIAN DAN PEMBAHASAN}
\hspace{1cm}Bagian ini berisi pembahasan temuan hasil penelitian dan sintesis temuan hasil penelitian.

\subsection*{Temuan Hasil Penelitian}
\hspace{1cm}Temuan hasil penelitian dikelompokkan dalam triad klasifikasi, yaitu peran penting
\textit{big data analytics} dalam pendeteksian \textit{financial fraud} di lembaga keuangan,
serta peran penting \textit{machine learning} dalam pendeteksian \textit{financial fraud} di lembaga keuangan,
serta peran penting \textit{artificial intelligence} dalam pendeteksian \textit{financial fraud} di lembaga keuangan.

\begin{center}
    \renewcommand{\arraystretch}{1.5}
    \subsubsection*{Tabel 3}
    \subsubsection*{Peran Penting \textit{Machine Learning} dalam Pendeteksian \textit{Financial Fraud} di Lembaga Keuangan}
    \begin{longtable}{|p{0.5cm}|p{3cm}|p{2.5cm}|p{5cm}|}
        \hline
        No & Judul Artikel & Peneliti & Temuan \\
        \hline
        1 & \textit{Auto Insurance Fraud Detection with Multimodal Learning} &
        <Riyan> & 
        Kerangka kerja \textit{auto insurance multi-modal learning}
        (AIML), yang meingintegrasikan pemrosesan bahasa alami teknik visi 
        komputer meningkatkan deteksi kecurangan  secara  signifikan  dengan memanfaatkan data \textit{multi-modal}
        untuk  prediksi  perilaku  kecurangan yang lebih akurat \\
        \hline
        2 & \textit{Automatic  Machine learning  Algorithms for  Fraud  Detection in  Digital  Payment Systems} &
        <Riyan> & 
        Penggunaan algoritma pembelajaran mesin otomatis dalam sistem pembayaran digital menunjukkan efektivitas
        tidak hanya melalui kualitas pengklsifikasian dan pengurangan biaya pengembangan, tetapi
        juga melalui peningkatan potensi interpretabilitas, yang sangat diperlukan untuk mengatasi tantangan yang berkembang
        dalam mendeteksi transaksi kecurangan. \\
        \hline
        3 & \textit{Chinese  Corporate Fraud Risk Assessment    with Machine Learning} &
        <Riyan> &
        Memanfaatkan berbagai algoritma \textit{machine learning} untuk meningkatkan akurasi model dan 
        \textit{area under the curve} (AUC) dan \textit{receiver operating characteristic} (ROC) adalah kunci
        dalam deteksi kecurangan, dengan sumber data \textit{real-time} dan analisis sentimen dari berita atau
        media sosial yang memperkuat kerangka kerja melawan taktik kecurangan yang terus berkembang. \\
        \hline
        4 & \textit{Detecting Fraud Transaction using Ripper Algorithm Combines with Ensemble  Learning Mode} &
        <Riyan> &
        Penggabungan algoritma \textit{Ripper} dengan model \textit{ensemble learning} terutama
        menggunakan \textit{gradient boosting}, menghasilkan akurasi deteksi lebih dari 99\%,
        menunjukkan efektivitas pembelajaran mesin dalam meningkatkan kemampuan deteksi kecurangan. \\
        \hline
        5 & \textit{Detecting Insurance Fraud Using Supervised and Unsupervised Machine Learning} &
        <Riyan> &
        Studi ini mengevaluasi metode \textit{unsupervised} dan \textit{supervised machine learning} untuk deteksi 
        \textit{insurance fraud} menggunakan data klaim ekslusif, menemukan bahwa \textit{isolation forest} dalam
        \textit{unsupervised learning} sangat efektif, sementara metode \textit{supervised learning}
        juga berkinerja baik meskipun dengan jumlah kasus kecurangan berlabel terbatas.\\
        \hline
        6 & \textit{Federated  Learning Model  for  Credit Card Fraud Detection  with  Data Balancing Techniques} &
        <Riyan> &
        \textit{Federated learning model} meningkatkan deteksi kecurangan kartu kredit dengan memungkinkan kolaborasi
        antar bank tanpa berbagi dats sensitif, manajemen ketidakseimbangan kelas melalui \textit{resampling}
        hibrida meingkatkan kinerja model klasifikasi, dan pengklsifikasi \textit{random forest}
        menonjol sebagai metode terbaik berdasarkan parameter kinerja dalam mendeteksi transaksi \textit{fraud}.\\
        \hline
    \end{longtable}
\end{center}

\begin{flushleft}
Sumber: Data Diolah, 2024
\end{flushleft}

Berikut merupakan kesimpulan mengapa \textit{machine learning} berperan penting dalam deteksi
\textit{financial fraud} menurut hasil penelitian yang telah dijabarkan di atas.

\begin{enumerate}
    \setlength{\itemsep}{0pt}
    \item Data \textit{multi-modal} meningkatkan  akurasi  deteksi  kecurangan,  penggunaan  data  teks  dan  visual bersama-sama  
    memperkaya  informasi  yang  dimiliki  model,  sehingga  dapat  memprediksi  perilaku curang dengan lebih baik.
    \item Algoritma \textit{automated machine learning} dapat menyintesis model deteksi kecurangan dengan lebih
    efisien dan mengurangi biaya pengembangan, sambil menjaga kualitas hasil deteksi.
    \item Pemanfaatan beragam algoritma \textit{machine learning} dan sumber data \textit{real-time}
    dapat memperkuat model deteksi kecurangan terhadap taktik kecurangan yang terus berkembang.
    \item Model \textit{machine learning} seperti \textit{random forest} yang dioptimalkan dengan teknik
    \textit{oversampling} dapat unggul dalam mendeteksi kecurangan akuntasi, menunjukan pentingnya pemilihan
    dan optimasi model.
    \item Metode \textit{federated learning} memungkinkan kolaborasi antar bank tanpa mengorbankan privasi data,
    meningkatkan deteksi kecurangan kartu kredit dengan mengumpulkan lebih banyak data untuk pelatihan model.
\end{enumerate}

\subsection*{Sintesis Temuan Hasil Penelitian}
\hspace{1cm} Bagian ini akan menyajikan sintesis dari 20 artikel temuan, yang secara khusus dirancang
untuk menjawab pertanyaan penelitian atau \textit{research question} yang telah dirumuskan sebelumnya.

\subsection*{Kekuatan dan Kelamahan \textit{Big Data Analytics, Machine Learning,} dan \textit{Artificial Intelligence} dalam Pendeteksian \textit{Financial Fraud} di Lembaga Keuangan}
\hspace{1cm}Untuk mengetahui apakah \textit{big data analytics, machine learning,} dan \textit{artificial intelligence}
berperan penting dalam pendeteksian \textit{financial fraud} di lembaga keuangan, maka dibutuhkan penjabaran tentang kekuatan dan kelemahan masing-masing
variabel tersebut dalam pendeteksian \textit{financial fraud} di lembaga keuangan.

\begin{center}
    \renewcommand{\arraystretch}{1.5}
    \subsection*{Tabel 6}
    \subsection*{Persentase Penilaian \textit{Big Data Analytics} dalam Pendeteksian \textit{Financial Fraud} di Lembaga Keuangan}
    \begin{longtable}{|l|c|c|}
        \hline
        \textbf{Kategori} & \textbf{Kekuatan (\%)} & \textbf{Kelemahan (\%)} \\
        \hline
        Efektivitas Deteksi & 90 & 10 \\
        \hline
        Akurasi & 85 & 15 \\
        \hline
        Kecepatan & 80 & 20 \\
        \hline
        Kemudahan Penggunaan & 70 & 30 \\
        \hline
        Biaya & 60 & 40 \\
        \hline
        Adaptabilitas & 75 & 25 \\
        \hline
    \end{longtable}
\end{center}

\begin{flushleft}
Sumber: Data Diolah, 2024
\end{flushleft}

\subsection*{Perkembangan Penelitian BDA, ML, dan AI dalam Pendeteksian \textit{Financial Fraud}}
\hspace{1cm}Penelitian tentang analisis \textit{big data} dalam deteksi kecurangan telah berkembang pesat sejak awal abad ke-21, didorong oleh krisis keuangan tahun 2008 dan regulasi yang memaksa sektor perbankan untuk lebih transparan serta memanfaatkan data pelanggan secara luas. Deteksi kecurangan diidentifikasi sebagai aplikasi penting \textit{big data analytics} (BDA) dalam perbankan, membantu meningkatkan pengambilan keputusan dan manajemen risiko. 
Penelitian \textit{machine learning} dalam deteksi kecurangan telah berkembang dari model tunggal ke model ansambel untuk meningkatkan akurasi dan efisiensi, dengan pengenalan algoritma seperti regresi logistik, SVM, dan \textit{XGBoost}. Peran AI juga telah berkembang untuk memerangi ancaman \textit{cyber} yang kompleks seperti pencurian identitas dan \textit{malware}, serta meningkatkan langkah-langkah keamanan secara keseluruhan. 
Prospek masa depan AI dalam deteksi kecurangan sangat menjanjikan, dengan kemajuan berkelanjutan yang diharapkan dapat memperkuat langkah-langkah keamanan siber secara efektif.

\subsection*{Ekplorasi Arah Penelitian Masa Depan}
\hspace{1cm}Dari pembahasan sintesis temuan hasil penelitian, dapat terlihat bahwa \textit{big data analytics} dan \textit{artificial intelligence} sudah cukup efektif dalam mendeteksi \textit{financial fraud} menurut skor penilaian. 
Sedangkan, beberapa algoritma pada \textit{machine learning} memiliki skor kelemahan yang lumayan besar, contohnya adalah \textit{federated learning} dan jaringan \textit{neural grafik}, khususnya model HAN. Untuk itu, ada beberapa integrasi antar algoritma yang dapat dilakukan untuk mengatasi kelemahan yang ada pada kedua algoritma tersebut.

\section*{Kesimpulan dan Rekomendasi}
\hspace{1cm}Bagian ini berisi kesimpulan dan rekomendasi untuk penelitian selanjutnya.

\subsection*{Kesimpulan}
\hspace{1cm}Kesimpulan dari penelitian ini adalah penerapan \textit{big data analytics} (BDA), \textit{machine learning} (ML), dan \textit{artificial intelligence} (AI) sangat berperan penting dalam pendeteksian \textit{financial fraud}, 
dilihat dari rata-rata kekuatan masing-masing teknologi tersebut yang menunjukkan kekuatan yang signifikan. 
Sistem deteksi \textit{financial fraud} dengan menggunakan BDA dan AI masing-masing menunjukkan rata-rata kekuatan yang sama, yaitu sebesar 76.67\%. Di sisi lain, dalam konteks pendeteksian \textit{financial fraud} menggunakan \textit{machine learning}, terdapat beberapa algoritma yang memiliki skor kelemahan yang tinggi. 
Algoritma tersebut adalah \textit{GNN} dan \textit{federated learning}. 
Untuk mengatasi kelemahan tersebut, dibutuhkan pengintegrasian dengan algoritma yang memiliki skor kekuatan yang tinggi. Oleh karena itu, diperlukan penelitian tentang integrasi algoritma \textit{machine learning} di masa mendatang.

\subsection*{Rekomendasi}
\hspace{1cm}Rekomendasi yang ditujukan pada peneliti yang akan menggarap penelitian sejenis adalah perlunya indikator yang dapat menjadi acuan untuk penilaian variabel, karena penilaian indikator dalam penelitian ini masih dilakukan secara subjektif. 
Topik yang dapat digunakan untuk penelitian lebih lanjut adalah mengenai integrasi algoritma \textit{machine learning} yang kurang efektif dengan algoritma yang memiliki efektivitas tinggi. Tujuannya adalah untuk memperkuat efektivitas, akurasi, dan berbagai aspek lain dari deteksi \textit{financial fraud} di lembaga keuangan. 
Salah satu contoh penelitian yang bisa dieksplorasi di masa mendatang adalah penggabungan atau integrasi algoritma \textit{graph neural networks} (GNN) dengan \textit{random forest}, serta kombinasi \textit{federated learning} dengan \textit{gradient boosting}. Untuk membuktikan hasil dari integrasi tersebut, maka perlu dilakukan penelitian secara empiris.

\end{document}